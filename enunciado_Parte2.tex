% Options for packages loaded elsewhere
\PassOptionsToPackage{unicode}{hyperref}
\PassOptionsToPackage{hyphens}{url}
\PassOptionsToPackage{dvipsnames,svgnames*,x11names*}{xcolor}
%
\documentclass[
]{article}
\usepackage{amsmath,amssymb}
\usepackage{lmodern}
\usepackage{ifxetex,ifluatex}
\ifnum 0\ifxetex 1\fi\ifluatex 1\fi=0 % if pdftex
  \usepackage[T1]{fontenc}
  \usepackage[utf8]{inputenc}
  \usepackage{textcomp} % provide euro and other symbols
\else % if luatex or xetex
  \usepackage{unicode-math}
  \defaultfontfeatures{Scale=MatchLowercase}
  \defaultfontfeatures[\rmfamily]{Ligatures=TeX,Scale=1}
\fi
% Use upquote if available, for straight quotes in verbatim environments
\IfFileExists{upquote.sty}{\usepackage{upquote}}{}
\IfFileExists{microtype.sty}{% use microtype if available
  \usepackage[]{microtype}
  \UseMicrotypeSet[protrusion]{basicmath} % disable protrusion for tt fonts
}{}
\makeatletter
\@ifundefined{KOMAClassName}{% if non-KOMA class
  \IfFileExists{parskip.sty}{%
    \usepackage{parskip}
  }{% else
    \setlength{\parindent}{0pt}
    \setlength{\parskip}{6pt plus 2pt minus 1pt}}
}{% if KOMA class
  \KOMAoptions{parskip=half}}
\makeatother
\usepackage{xcolor}
\IfFileExists{xurl.sty}{\usepackage{xurl}}{} % add URL line breaks if available
\IfFileExists{bookmark.sty}{\usepackage{bookmark}}{\usepackage{hyperref}}
\hypersetup{
  pdftitle={Proyecto asignatura Estadística (MAT3) GIN2 Grupos 1 y 3: PARTE 2. CUrso 2020-2021},
  pdfauthor={Antoni, Pol Martínez; Josep, Oliver Vallespir; Gabriel, Riutort Álvarez; Jorge, Oliveras Hidalgo},
  colorlinks=true,
  linkcolor=Maroon,
  filecolor=Maroon,
  citecolor=Blue,
  urlcolor=blue,
  pdfcreator={LaTeX via pandoc}}
\urlstyle{same} % disable monospaced font for URLs
\usepackage[margin=1in]{geometry}
\usepackage{graphicx}
\makeatletter
\def\maxwidth{\ifdim\Gin@nat@width>\linewidth\linewidth\else\Gin@nat@width\fi}
\def\maxheight{\ifdim\Gin@nat@height>\textheight\textheight\else\Gin@nat@height\fi}
\makeatother
% Scale images if necessary, so that they will not overflow the page
% margins by default, and it is still possible to overwrite the defaults
% using explicit options in \includegraphics[width, height, ...]{}
\setkeys{Gin}{width=\maxwidth,height=\maxheight,keepaspectratio}
% Set default figure placement to htbp
\makeatletter
\def\fps@figure{htbp}
\makeatother
\setlength{\emergencystretch}{3em} % prevent overfull lines
\providecommand{\tightlist}{%
  \setlength{\itemsep}{0pt}\setlength{\parskip}{0pt}}
\setcounter{secnumdepth}{5}
\renewcommand{\contentsname}{Contenidos}
\ifluatex
  \usepackage{selnolig}  % disable illegal ligatures
\fi

\title{Proyecto asignatura Estadística (MAT3) GIN2 Grupos 1 y 3: PARTE
2. CUrso 2020-2021}
\author{Antoni, Pol Martínez; Josep, Oliver Vallespir; Gabriel, Riutort
Álvarez; Jorge, Oliveras Hidalgo}
\date{15/06/2021}

\begin{document}
\maketitle

{
\hypersetup{linkcolor=blue}
\setcounter{tocdepth}{2}
\tableofcontents
}
\hypertarget{parte-2-estaduxedstica-inferencial}{%
\section{Parte 2: Estadística
Inferencial}\label{parte-2-estaduxedstica-inferencial}}

A mi grupo le ha tocado de la web
\href{http://insideairbnb.com/get-the-data.html}{web de Airbnb}, Madrid,
Ciudad de Madrid, España.

Supongamos que los datos de la ciudad que ha sido asignada a cada grupo
corresponden a una muestra aleatoria simple de todas las viviendas que
se podrían alquilar en la ciudad. Utilizando esta muestra se pide:

\hypertarget{soliciuxf3n-ejercicio-1}{%
\subsection{Solición ejercicio 1}\label{soliciuxf3n-ejercicio-1}}

\begin{enumerate}
\def\labelenumi{\arabic{enumi}.}
\tightlist
\item
  Calcular una estimación puntual de la media para la variable
  \texttt{price} y el error estándar del estimador.
\end{enumerate}

\hypertarget{soliciuxf3n-ejercicio-2}{%
\subsection{Solición ejercicio 2}\label{soliciuxf3n-ejercicio-2}}

\begin{enumerate}
\def\labelenumi{\arabic{enumi}.}
\setcounter{enumi}{1}
\tightlist
\item
  Calcular un intervalo de confianza, al nivel de confianza del 95\%,
  para la variable \texttt{price}.
\end{enumerate}

\hypertarget{soliciuxf3n-ejercicio-3}{%
\subsection{Solición ejercicio 3}\label{soliciuxf3n-ejercicio-3}}

\begin{enumerate}
\def\labelenumi{\arabic{enumi}.}
\setcounter{enumi}{2}
\tightlist
\item
  Calcular un intervalo de confianza, al nivel de confianza del 99\%,
  para la proporción de viviendas que tienen un
  \texttt{review\_scores\_rating} inferior a 95\%.
\end{enumerate}

\hypertarget{soliciuxf3n-ejercicio-4}{%
\subsection{Solición ejercicio 4}\label{soliciuxf3n-ejercicio-4}}

\begin{enumerate}
\def\labelenumi{\arabic{enumi}.}
\setcounter{enumi}{3}
\tightlist
\item
  Supongamos que un responsable de Airbnb asegura que la media de los
  valores de \texttt{review\_scores\_rating} de las viviendas de su
  portal es superior a 95. Contrastad esta hipótesis.
\end{enumerate}

\hypertarget{soliciuxf3n-ejercicio-5}{%
\subsection{Solición ejercicio 5}\label{soliciuxf3n-ejercicio-5}}

\begin{enumerate}
\def\labelenumi{\arabic{enumi}.}
\setcounter{enumi}{4}
\tightlist
\item
  Calcular el intervalo de confianza, con un nivel de confianza del
  95\%, asociado al contraste del ejercicio anterior.
\end{enumerate}

\hypertarget{soliciuxf3n-ejercicio-6}{%
\subsection{Solición ejercicio 6}\label{soliciuxf3n-ejercicio-6}}

\begin{enumerate}
\def\labelenumi{\arabic{enumi}.}
\setcounter{enumi}{5}
\tightlist
\item
  Considera ahora los datos de \texttt{price} para la ciudad de New York
  del mes de febrero de 2020 ( están en
  \url{http://insideairbnb.com/get-the-data.html}, y debe pulsar en
  `show archived fecha ). Compararemos los valores de esta variable con
  los correspondientes a la ciudad que tiene asignada. Haga un contraste
  de hipótesis para decidir si las desviaciones típicas de los precios
  de las dos ciudades son iguales o diferentes. Considera que las
  distribuciones de los valores de precio en las poblaciones son
  normales.
\end{enumerate}

\hypertarget{soliciuxf3n-ejercicio-7}{%
\subsection{Solición ejercicio 7}\label{soliciuxf3n-ejercicio-7}}

\begin{enumerate}
\def\labelenumi{\arabic{enumi}.}
\setcounter{enumi}{6}
\tightlist
\item
  A partir de los resultados del apartado anterior contratad la
  hipótesis de que los precios medios en las dos ciudades son iguales.
\end{enumerate}

\hypertarget{soliciuxf3n-ejercicio-8}{%
\subsection{Solición ejercicio 8}\label{soliciuxf3n-ejercicio-8}}

\begin{enumerate}
\def\labelenumi{\arabic{enumi}.}
\setcounter{enumi}{7}
\tightlist
\item
  Utilice el test de Kolmogorov-Smirnov-Lilliefors para confirmar o
  rechazar la hipótesis de que la distribución de los valores de la
  variable \texttt{price} es normal, decidid el resultado del contraste
  con el \(p\)-valor.
\end{enumerate}

\hypertarget{soliciuxf3n-ejercicio-9}{%
\subsection{Solición ejercicio 9}\label{soliciuxf3n-ejercicio-9}}

\begin{enumerate}
\def\labelenumi{\arabic{enumi}.}
\setcounter{enumi}{8}
\tightlist
\item
  Analizad la dependencia entre las variables \texttt{Price} y
  \texttt{review\_scores\_rating} de la ciudad que tiene asignada.
  Seguid los siguientes pasos:
\end{enumerate}

\hypertarget{a}{%
\subsubsection{a)}\label{a}}

\begin{verbatim}
* a) Seleccione del data frame las muestras que tienen valores diferentes de NA por las dos
\end{verbatim}

variables.

\hypertarget{b}{%
\subsubsection{b)}\label{b}}

\begin{verbatim}
* b) A continuación agrupau los valores de cada variable utilizando los intervalos siguientes: $[ \min, Q_1), [Q_1, Q_2), [Q_2, Q_3)$ y $[Q_3, \max]$.
\end{verbatim}

Los valores \(\min\) y \(\max\) son el mínimo y el máximo de la
variable, respectivamente. Mientras que \(Q_1\), \(Q_2\) y \(Q_3\)·
representan los cuartiles primero, segundo (mediana) y tercero. Si los
valor mínimo y máximo de algún intervalo son iguales elimine este
intervalo.

\hypertarget{c}{%
\subsubsection{c)}\label{c}}

\begin{verbatim}
* c) Organizad los datos agrupados en intervalos en una tabla de contingencia `Price` versus `review_scores_rating` .
\end{verbatim}

\hypertarget{d}{%
\subsubsection{d)}\label{d}}

\begin{verbatim}
* d) A partir de esta tabla haced un test $\chi^2$ de independencia para determinar si las dos variables son independientes, con un nivel de significación del 0.05.
\end{verbatim}

\end{document}
